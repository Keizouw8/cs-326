\documentclass[letterpaper]{article}
\usepackage{titlesec}
\usepackage{geometry}
\usepackage{mathtools}
\usepackage{amssymb}
\usepackage{amsmath}
\usepackage{listings}
\usepackage{changepage}
\usepackage{color}
\usepackage{qtree}
\usepackage{fancyvrb}
\usepackage{enumitem}

\definecolor{dkgreen}{rgb}{0,0.6,0}
\definecolor{mauve}{rgb}{0.58,0,0.82}

\title{Homework 5}
\author{Keizou Wang}
\date{April 10, 2025}

\titleformat{\section}{\normalfont\normalsize}{\bfseries\thesection.}{1em}{}
\titleformat{\subsection}{\normalfont\normalsize}{(\textit{\alph{subsection}})}{1em}{}

\lstset{
	language=C++,
	basicstyle={\ttfamily},
	keywordstyle=\color{blue},
	commentstyle=\color{dkgreen},
	stringstyle=\color{mauve},
	tabsize=4,
	showstringspaces=false
}

\begin{document}

\maketitle

\section{Suppose we are compiling for a machine with 1-byte characters, 2-byte shorts, 4- byte integers, and 8-byte reals, and with alignment rules that require the address of every primitive data element to be a multiple of the element's size. Suppose further that the compiler is not permitted to reorder fields. How much space will be consumed by the following array? Explain.}
\begin{adjustwidth}{2.35em}{0pt}
\begin{Verbatim}[tabsize=4]
A : array [0..9] of record
	s : short
	c : char
	t : short
	d : char
	r : real
	i : integer
\end{Verbatim}
Each record has a size of $2+1+2+1+8+4=18$ bytes. Each index must align to a multiple of 4 bytes, so each is 20 bytes. $20\cdot10=200$ bytes.
\end{adjustwidth}

\section{For the following code specify which of the variables a,b,c,d are type equivalent.}
\begin{adjustwidth}{2.35em}{0pt}
\begin{Verbatim}[tabsize=4]
Type T = array [1..10] of integer
     S = T
a : T
b : T
c : S
d : array [1..10] of integer
\end{Verbatim}
\end{adjustwidth}
\subsection{Under structural equivalence.}
\begin{adjustwidth}{2.35em}{0pt}
\begin{Verbatim}[tabsize=4]
a=b=c=d 
\end{Verbatim}
\end{adjustwidth}
\subsection{Under strict name equivalence.}
\begin{adjustwidth}{2.35em}{0pt}
\begin{Verbatim}[tabsize=4]
a=b, c, d
\end{Verbatim}
\end{adjustwidth}
\subsection{Under loose name equivalence.}
\begin{adjustwidth}{2.35em}{0pt}
\begin{Verbatim}[tabsize=4]
a=b=c, d
\end{Verbatim}
\end{adjustwidth}

\section{We are trying to run the following C program:}
\begin{adjustwidth}{2.35em}{0pt}
\begin{lstlisting}
typedef struct
{
	int a;
	char * b;
} Cell;

void AllocateCell (Cell * q)
{
	q = (Cell *) malloc ( sizeof(Cell) );
}

void main ()
{
	Cell * c;
	AllocateCell (c);
	c->a = 1;
	free(c);
}
\end{lstlisting}
\end{adjustwidth}
\subsection{The program produces a run-time error. Why?}
\begin{adjustwidth}{2.35em}{0pt}
	The programing produces a run-time error because the initialization and assignment to q on line 9 in the Allocate Cell function does not modify c. Therefore, c remains unintialized when the free function is called on it, causing the error.
\end{adjustwidth}
\subsection{Rewrite the functions \texttt{AllocateCell} and \texttt{main} so that the program runs correctly.}
\begin{adjustwidth}{2.35em}{0pt}
\begin{lstlisting}
typedef struct{
	int a;
	char * b;
} Cell;

void AllocateCell (Cell*& q){
	q = (Cell *) malloc ( sizeof(Cell) );
}

void main (){
	Cell* c;
	AllocateCell (c);
	c->a = 1;
	free(c);
}
\end{lstlisting}
\end{adjustwidth}

\section{Consider the following C declaration, compiled on a 32-bit Pentium machine.}
\begin{adjustwidth}{2.35em}{0pt}
\begin{lstlisting}
struct
{
	int n;
	char c;
} A[10][10];
\end{lstlisting}
If the address of \texttt{A[0][0]} is 1000 (decimal), what is the address of \texttt{A[3][7]}? Explain how this is computed.
\end{adjustwidth}
\begin{adjustwidth}{1.8em}{0pt}
\begin{tabular}{l c l}
Size(int)=4, Size(char)=1 & $\Rightarrow$ & Size(Elem)=5 \\
Size(Elem)=5, Size(Index)=4$k$: $k\in\mathbb{N}$  & $\Rightarrow$ & Size(Index)=8 \\
Columns=10, x=3, y=7  & $\Rightarrow$ & Index=x(Columns)+y=37 \\
Address = 1000 + Index $\cdot$ Size(Index) = 1296
\end{tabular}
\end{adjustwidth}

\section{Write a small fragment of code that shows how unions can be used in C to interpret the bits of a value of one type as if they represented a value of some other type (non-converting type cast).}
\begin{adjustwidth}{2.35em}{0pt}
\begin{lstlisting}
union Number{
	int i;
	float f;
};

int main(){
	int original = 37;
	Number originalUnion = { .i=original };
	float unionInterpretedFloat = originalUnion.f; // 5.1848e-44
}
\end{lstlisting}
\end{adjustwidth}

\end{document}
