\documentclass[letterpaper]{article}
\usepackage{titlesec}
\usepackage{geometry}
\usepackage{mathtools}
\usepackage{amssymb}
\usepackage{amsmath}
\usepackage{listings}
\usepackage{changepage}
\usepackage{color}
\usepackage{qtree}
\usepackage{fancyvrb}
\usepackage{enumitem}

\definecolor{dkgreen}{rgb}{0,0.6,0}
\definecolor{mauve}{rgb}{0.58,0,0.82}

\title{Homework 3}
\author{Keizou Wang}
\date{February 27, 2025}

\titleformat{\section}{\normalfont\normalsize}{\bfseries\thesection.}{1em}{}
\titleformat{\subsection}{\normalfont\normalsize}{(\textit{\alph{subsection}})}{1em}{}

\lstset{
	language=C++,
	basicstyle={\ttfamily},
	keywordstyle=\color{blue},
	commentstyle=\color{dkgreen},
	stringstyle=\color{mauve},
	tabsize=4,
	showstringspaces=false
}

\begin{document}

\maketitle

\section{Assume that each integer variable occupies four bytes. How much total space is required for
the variables in this code? Justify your answer.}
\begin{adjustwidth}{2.35em}{0pt}
\begin{lstlisting}
{ int a, b, c;
	...
	{ int d, e;
		...
		{ int f;
			...
		}
		...
	}
	...
	{ int g, h, i;
		...
	}
	...
}
\end{lstlisting}
The variables will occupy 24 bytes total because up to 6 variables exist in a given scope at a time.
\end{adjustwidth}

\section{Assuming static scope, what is the referencing environment (i.e., what names are known, and
what do they refer to) at the location marked by (*)?}
\begin{adjustwidth}{2.35em}{0pt}
\begin{Verbatim}[tabsize=4]
procedure P (A,B: real)
	X: real
	procedure Q (B,C: real)
		Y: real
		...
	procedure R (A,C: real)
		Z: real
		... // (*)
	...
\end{Verbatim}
A, B, C, X, and Z are known. B refers to P's parameter. X refers to the variable defined in P. A, C refer to R's parameters. Z refers to the variable defined in R.
\end{adjustwidth}

\section{Consider the following pseudocode:}
\begin{adjustwidth}{2.35em}{0pt}
\begin{Verbatim}[tabsize=4]
1. procedure main
2.   a: integer := 1
3.   b: integer := 2

4.   procedure middle
5.     b: integer := a

6.     procedure inner
7.       print a, b

8.     a: integer := 3

9.     // body of middle
10.    inner()
11.    print a, b

12.  // body of main
13.  middle()
14.  print a, b
\end{Verbatim}
\end{adjustwidth}
\subsection{Suppose this was code for a language with the declaration-order rules of C (but with nested subroutines). At each print statement, indicate which declarations of a and b are in the referencing environment. What does the program print?}
\begin{adjustwidth}{2.35em}{0pt}
Line 7
\begin{enumerate}[label=\alph*]
\item defined at line 2
\item defined at line 5
\end{enumerate}
Line 11
\begin{enumerate}[label=\alph*]
\item defined at line 8
\item defined at line 5
\end{enumerate}
Line 14
\begin{enumerate}[label=\alph*]
\item defined at line 2
\item defined at line 3
\end{enumerate}
Program output:
\begin{Verbatim}
1 1
3 1
1 2
\end{Verbatim}
\end{adjustwidth}
\subsection{Repeat the exercise for the declaration-order rules of C\#.}
\begin{adjustwidth}{2.35em}{0pt}
Line 7
\begin{enumerate}[label=\alph*]
\item defined at line 8
\item defined at line 5
\end{enumerate}
Line 11
\begin{enumerate}[label=\alph*]
\item defined at line 8
\item defined at line 5
\end{enumerate}
Line 14
\begin{enumerate}[label=\alph*]
\item defined at line 2
\item defined at line 3
\end{enumerate}
Program returns error because a is used before definition at line 5.
\end{adjustwidth}
\subsection{Repeat the exercise for the declaration-order rules Modula-3.}
\begin{adjustwidth}{2.35em}{0pt}
Line 7
\begin{enumerate}[label=\alph*]
\item defined at line 8
\item defined at line 5
\end{enumerate}
Line 11
\begin{enumerate}[label=\alph*]
\item defined at line 8
\item defined at line 5
\end{enumerate}
Line 14
\begin{enumerate}[label=\alph*]
\item defined at line 2
\item defined at line 3
\end{enumerate}
Program output:
\begin{Verbatim}
3 3
3 3
1 2
\end{Verbatim}
\end{adjustwidth}

\section{Consider the following pseudocode:}
\begin{adjustwidth}{2.35em}{0pt}
\begin{Verbatim}[tabsize=4]
x: integer := 1
y: integer := 2
procedure add
	x := x + y
procedure second (P: procedure)
	x: integer := 2
	P()
procedure first
	y: integer := 3
	second(add)
// main program
first()
write integer(x)
\end{Verbatim}
\end{adjustwidth}
\subsection{What does this program print if the language uses static scoping?}
\begin{adjustwidth}{2.35em}{0pt}
\begin{Verbatim}[tabsize=4]
3
\end{Verbatim}
\end{adjustwidth}
\subsection{What does it print if the language uses dynamic scoping with deep binding?}
\begin{adjustwidth}{2.35em}{0pt}
\begin{Verbatim}[tabsize=4]
4
\end{Verbatim}
\end{adjustwidth}
\subsection{What does it print if the language uses dynamic scoping with shallow binding?}
\begin{adjustwidth}{2.35em}{0pt}
\begin{Verbatim}[tabsize=4]
1
\end{Verbatim}
\end{adjustwidth}
\end{document}
