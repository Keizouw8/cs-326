\documentclass[letterpaper]{article}
\usepackage{titlesec}
\usepackage{geometry}
\usepackage{mathtools}
\usepackage{amssymb}
\usepackage{amsmath}
\usepackage{listings}
\usepackage{changepage}
\usepackage{color}
\usepackage{qtree}
\usepackage{fancyvrb}
\usepackage{enumitem}

\definecolor{dkgreen}{rgb}{0,0.6,0}
\definecolor{mauve}{rgb}{0.58,0,0.82}

\title{Homework 6}
\author{Keizou Wang}
\date{April 22, 2025}

\titleformat{\section}{\normalfont\normalsize}{\bfseries\thesection.}{1em}{}
\titleformat{\subsection}{\normalfont\normalsize}{(\textit{\alph{subsection}})}{1em}{}

\lstset{
	language=C++,
	basicstyle={\ttfamily},
	keywordstyle=\color{blue},
	commentstyle=\color{dkgreen},
	stringstyle=\color{mauve},
	tabsize=4,
	showstringspaces=false
}

\begin{document}

\maketitle

\section{Consider the following C program:}
\begin{adjustwidth}{2.35em}{0pt}
\begin{lstlisting}
void foo()
{
	int i;
	printf("%d ", i++);
}

void main()
{
	int j;
	for (j = 1; j <= 10; j++)
		foo();
}
\end{lstlisting}
Local variable i in subroutine foo is never initialized. On many systems, however, variable i appears to “remember” its value between the calls to foo, and the program will print 0 1 2 3 4 5 6 7 8 9.
\end{adjustwidth}
\subsection{Suggest an explanation for this behavior}
\begin{adjustwidth}{2.35em}{0pt}
i is declared without a default value every time so the value is determined by what's already at the address. In this scenario, it seems whenever foo is ran, i takes the same address and the same value is incremented.
\end{adjustwidth}
\subsection{Change the code above (without modifying function foo) to alter this behavior.}
\begin{adjustwidth}{2.35em}{0pt}
\begin{lstlisting}
void main()
{
	int j, i;
	for (j = 1; j <= 10; j++){
		printf("%d ", i);
		foo();
	}
}
\end{lstlisting}
\end{adjustwidth}

\section{Can you write a macro in C that “returns” the factorial of an integer argument (without calling a subroutine)? Why or why not?}
\begin{adjustwidth}{2.35em}{0pt}
It would be impossible because recursion is impossible with just textual substitution and a loop solution can't directly ``return" a value.
\end{adjustwidth}

\section{Consider a subroutine \texttt{swap} that takes two parameters and simply swaps their values. For example, after calling \texttt{swap(X,Y)}, \texttt{X} should have the original value of \texttt{Y} and \texttt{Y} the original value of \texttt{X}. Assume that variables to be swapped can be simple or subscripted (elements of an array), and they have the same type (integer). Show that it is \textit{impossible} to write such a general-purpose \texttt{swap} subroutine in a language with:}
\subsection{Parameters passing by value.}
\begin{adjustwidth}{2.35em}{0pt}
It's impossible when passing by value because the programmer only has access to the values of X and Y and not the addresses of the variables. Therefore, the programmer can't set the value at the address of X to the value of Y and vice versa.
\end{adjustwidth}
\subsection{Parameters passing by name.}
\begin{adjustwidth}{2.35em}{0pt}
Consider the case \texttt{swap(i, a[i])}. After changing the value of \texttt{i}, \texttt{a[i]} references a different address so the behavior becomes unpredictable.
\end{adjustwidth}

\section{Consider the following program, written in no particular language.}
\begin{adjustwidth}{2.35em}{0pt}
\begin{Verbatim}
procedure f (x, y, z)
	x := x + 1
	y := z
	z := z + 1

// main
i := 1;
a[1] := 10;
a[2] := 11
f (i, a[i], i);
print (i);
print (a[1]);
print (a[2]);
\end{Verbatim}
Show what the program prints in the case of parameter passing by
\end{adjustwidth}
\subsection{value}
\begin{adjustwidth}{2.35em}{0pt}
\begin{Verbatim}
1 10 11
\end{Verbatim}
\end{adjustwidth}
\subsection{reference}
\begin{adjustwidth}{2.35em}{0pt}
\begin{Verbatim}
3 2 11
\end{Verbatim}
\end{adjustwidth}
\subsection{value-result}
\begin{adjustwidth}{2.35em}{0pt}
If the value of \texttt{i} is updated to the result before \texttt{a[i]} is updated:
\begin{Verbatim}
2 10 1
\end{Verbatim}
If the value of \texttt{i} is updated to the result after \texttt{a[i]} is updated:
\begin{Verbatim}
2 1 11
\end{Verbatim}
\end{adjustwidth}
\subsection{name}
\begin{adjustwidth}{2.35em}{0pt}
\begin{Verbatim}
3 10 2
\end{Verbatim}
\end{adjustwidth}

\section{Does a program run faster when the programmer does not specify values for the optional parameters in a subroutine call?}
\begin{adjustwidth}{2.35em}{0pt}
No because the default value is set to the parameter in runtime so it's equivalent to setting the value within the function normally.
\end{adjustwidth}

\end{document}
