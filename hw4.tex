\documentclass[letterpaper]{article}
\usepackage{titlesec}
\usepackage{geometry}
\usepackage{mathtools}
\usepackage{amssymb}
\usepackage{amsmath}
\usepackage{listings}
\usepackage{changepage}
\usepackage{color}
\usepackage{qtree}
\usepackage{fancyvrb}
\usepackage{enumitem}

\definecolor{dkgreen}{rgb}{0,0.6,0}
\definecolor{mauve}{rgb}{0.58,0,0.82}

\title{Homework 4}
\author{Keizou Wang}
\date{March 18, 2025}

\titleformat{\section}{\normalfont\normalsize}{\bfseries\thesection.}{1em}{}
\titleformat{\subsection}{\normalfont\normalsize}{(\textit{\alph{subsection}})}{1em}{}

\lstset{
	language=C++,
	basicstyle={\ttfamily},
	keywordstyle=\color{blue},
	commentstyle=\color{dkgreen},
	stringstyle=\color{mauve},
	tabsize=4,
	showstringspaces=false
}

\begin{document}

\maketitle

\section{Translate the following expression into (a) postfix and (b) prefix notation:}
\begin{adjustwidth}{2.35em}{0pt}
\begin{Verbatim}
(b + sqrt(b*b - 4*a*c))/(2*a)
\end{Verbatim}
Prefix: \texttt{/ + b sqrt - *bb **4ac *2a} \\
Postfix: \texttt{b bb* 4a*c* - sqrt + 2a* /}
\end{adjustwidth}

\section{Some languages (e.g., Algol 68) do not employ short-circuit evaluation for Boolean expressions. However, in such languages an if…then…else construct (which only evaluates the arm that is needed) can be used as an expression that returns a value. Show how to use if…then…else to achieve the effect of short-circuit evaluation for A and B and for A or B.}
\begin{adjustwidth}{2.35em}{0pt}
\begin{Verbatim}[tabsize=4]
A and B: if A then B else False
A or B: if A then True else B
\end{Verbatim}
\end{adjustwidth}

\section{Consider a midtest loop, here written in C, that processes all lines in the input until a blank line is found:}
\begin{adjustwidth}{2.35em}{0pt}
\begin{lstlisting}
for ( ; ; ) {
	line = read_line();
	if (all_blanks(line)) break;
	process_line(line);
}
\end{lstlisting}
\end{adjustwidth}
\subsection{Show how you might accomplish the same task in C using a while loop.}
\begin{adjustwidth}{2.35em}{0pt}
\begin{lstlisting}
line = read_line();
while (!all_blanks(line)) {
	process_line(line);
	line = read_line();
}
\end{lstlisting}
\end{adjustwidth}
\subsection{Show how you might accomplish the same task in C using a do loop.}
\begin{adjustwidth}{2.35em}{0pt}
\begin{lstlisting}
do{
	line = read_line();
	if(!all_blanks(line)) process_line(line);
}while (!all_blanks(line))
\end{lstlisting}
\end{adjustwidth}

\section{Write a \textit{tail-recursive} function in Scheme to compute n factorial ($n! = 1\times2\times\dots\times n$). You will probably want to define a “helper” function, as discussed in the textbook.}
\begin{adjustwidth}{2.35em}{0pt}
\begin{Verbatim}[tabsize=4]
(define (factorial n)
	(define (helper a n)
		(if (equal? n 1)
			a
			(helper (* a n) (- n 1))
		)
	)
	(helper 1 n)
)
\end{Verbatim}
\end{adjustwidth}

\section{Give an example in C in which an in-line subroutine may be significantly faster than a functionally equivalent macro. Give another example in which the macro is likely to be faster. Hint: think about applicative versus normal-order evaluation of arguments.}
\begin{adjustwidth}{2.35em}{0pt}
\begin{lstlisting}
#include <stdio.h>
#define macroSum(x, y) (x + y)

int functionSum(int x, int y){
	return x + y;
}

int factorial(int x){
	return x == 1 ? x : x * factorial(x - 1);
}

int main(){
	// Function faster than macro
	printf("%i\n", functionSum(factorial(10), factorial(10)));
	printf("%i", macroSum(factorial(10), factorial(10)));

	// Macro faster than function
	printf("%i\n", functionSum(10, 10));
	printf("%i", macroSum(10, 10));
}
\end{lstlisting}
\end{adjustwidth}

\end{document}
